\documentclass[a4paper,12pt]{article}
\usepackage[T1]{fontenc}
\usepackage[utf8]{inputenc}
\usepackage{lmodern}
\usepackage[french]{babel}
\usepackage{url,csquotes}
\usepackage[hidelinks,hyperfootnotes=false]{hyperref}
%\usepackage[titlepage]{polytechnique}
%\usepackage[titlepage,fancysections,pagenumber]{polytechnique}
\usepackage{float}
\usepackage{graphicx}
\usepackage{subfig}
\usepackage[margin=2cm]{geometry}

\usepackage{tcolorbox}
\usepackage{amsmath}
\usepackage{hyperref}

\usepackage{titlesec}

% Define the font size for subsection headings

%\setcounter{tocdepth}{}
%\usepackage[charter]{mathdesign}

%\setcounter{secnumdepth}{0}


\title{PHY361 DM1 \\
 Oscillations d’atomes piégés dans un potentiel parabolique}
%\subtitle{Stage linguistique à l'IFLS \\
%Formation Préparatoire}
\author{Isai GORDEEV et Chich\\
Promotion X2022, section Escrime}



\begin{document}

\maketitle



\section{Mesure par vol libre de la densité de probabilité de l’impulsion}
$$\hat H =  \hbar\omega(\hat a^{\dagger }\hat a + \frac12)$$
\subsection{}
Les états d'énergie propres d'opérateur $\hat H$
\begin{equation}\label{key}
	\hat H |\psi_n\rangle = E_n |\psi_n\rangle
\end{equation}
$$E_n = \hbar\omega(n+\frac 1 2) $$






\begin{equation}
	\varphi^2(p, t) =\frac{1}{2}\left(1+(\epsilon p)^2 - 2\epsilon p \sin\left(\frac {wt} 2\right)\right) \varphi^2_0(p)
\end{equation}

\subsection{}

Nous voyons que dans le graphique c'est une fonction périodique qui fait des oscillation autour d'un point fixe, que correspond à la fonction obtenue dans (41). 

En moment de $\tau = 8\tau_0$ la courbe se met dans la position initiale, donc 

\begin{equation}\label{key}
	4w\tau_0 = 2\pi \rightarrow \tau_0 = \frac {1} {4\nu} = 2.7\mu s
\end{equation}



\end{document}